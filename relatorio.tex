%***************************************************
%* Língua Natural
%* Mini-Projeto 1
%***************************************************
\documentclass[12pt]{article}
\usepackage[utf8]{inputenc}
\usepackage{indentfirst}
\usepackage{graphicx}
\renewcommand{\familydefault}{\sfdefault}

\usepackage[letterpaper, portrait, margin=3cm]{geometry}

\begin{document}
\title{\vspace{-3cm}Língua Natural}
\author{Grupo 25 - Relatório do Mini-Projeto 1}
\date{}

\maketitle
83567 - Tiago Gonçalves

83576 - Vítor Nunes

\section*{Abordagem}
A Data foi vista como uma concatenação de três módulos. Conforme o exercício pedido escolhemos os módulos necessários reaproveitando os módulos já existentes.
\newline

A criação do FST \textsc{dia} consiste na união entre dois FST:
\begin{itemize}
	\item \textsc{1{\_}29nn} que contém a tradução de número para extenso de números entre 1 e 29 inclusive;
	\item \textsc{30{\_}31nn} que contém a tradução de número para extenso dos números 30 e 31;
\end{itemize}

O FST \textsc{pt2en} foi obitdo à custa da inversão, ou seja troca de \textit{input} por \textit{output}.

Assim é possível reaproveitar o \textsc{1{\_}29nn}, para os últimos dois dígitos do ano.

Assim a estrutura do \textsc{Dia} é:
$$ \textsc{Dia} = \textsc{1{\_}29nn} \cup \textsc{30{\_}31nn}  $$

Para o \textsc{Ano} a estrutura é:
$$ \textsc{Ano} = \textsc{Seculo} ^\frown \textsc{1{\_}99nn}  $$

O FST \textsc{seculo} contém o extenso dos primeiros dois digitos do ano: \textit{Dois mil e ...}

Na criação de \textsc{misto2texto} para o módulo mês fez-se uma composição entre \textsc{mmm2mm} (conversão de condensado para númerico) e \textsc{mes} (conversão de númerico para extenso) o que resultou numa conversão direta do formato condensado para extenso.




\end{document}

